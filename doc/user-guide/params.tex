\section{Parameter Descriptions}

Table~\ref{table:params} summarizes the key parameters for the \textbf{UART} module. Each can be customized during instantiation.

\renewcommand*{\arraystretch}{1.3}
\begingroup
\small
\rowcolors{2}{gray!30}{gray!10}
\arrayrulecolor{gray!80}

\begin{longtable}[H]{
  | p{0.22\textwidth}
  | p{0.13\textwidth}
  | p{0.08\textwidth}
  | p{0.08\textwidth}
  | p{0.42\textwidth} |
}
\hline
\rowcolor{gray}
\textcolor{white}{\textbf{Name}} &
\textcolor{white}{\textbf{Type}} &
\textcolor{white}{\textbf{Min}} &
\textcolor{white}{\textbf{Max}} &
\textcolor{white}{\textbf{Description}} \\ 
\hline
\endfirsthead

\hline
\rowcolor{gray}
\textcolor{white}{\textbf{Name}} &
\textcolor{white}{\textbf{Type}} &
\textcolor{white}{\textbf{Min}} &
\textcolor{white}{\textbf{Max}} &
\textcolor{white}{\textbf{Description}} \\ 
\hline
\endhead

\hline
\endfoot

\texttt{dataWidth} &
Integer &
8 &
32 &
Data width for APB bus operations. \\ \hline

\texttt{addressWidth} &
Integer &
1 &
-- &
Address width for APB bus. \\ \hline

\texttt{maxClocksPerBit} &
Integer &
2 &
-- &
Maximum allowed clock cycles per UART bit. Determines maximum baud rate. \\ \hline

\texttt{maxOutputBits} &
Integer &
5 &
16 &
Maximum number of data bits plus optional parity bit. Typically up to 9 or so, here parameterized. \\ \hline

\texttt{fifoDepth} &
Integer &
1 &
-- &
Depth of the internal RX/TX FIFOs. Must be a power of 2. \\ \hline

\texttt{syncDepth} &
Integer &
2 &
-- &
Depth for RX input synchronization. Recommended at least 2 for metastability protection. \\ \hline

\texttt{parity} &
Bool &
N/A &
N/A &
Default parity selection (odd/even). This can be overridden by registers at runtime. \\ \hline

\texttt{verbose} &
Bool &
N/A &
N/A &
Enables debugging \texttt{printf} statements. \\ \hline

\end{longtable}
\captionof{table}{UART Parameter Descriptions}
\label{table:params}
\endgroup

\noindent
A typical instantiation in Scala might look like:
\begin{lstlisting}[language=Scala]
// Example instantiation
val myUart = Module(new Uart(
  UartParams(
    dataWidth       = 32,
    addressWidth    = 32,
    maxClocksPerBit = 217,
    maxOutputBits   = 8,
    fifoDepth       = 16,
    syncDepth       = 2,
    parity          = false,
    verbose         = false
  ),
  formal = false
))
\end{lstlisting}
