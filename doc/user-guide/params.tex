\section{Parameter Descriptions}

The parameters for the \textbf{UART} module are listed in Table~\ref{table:uart_params}. These parameters configure the APB interface, baud rate generation, data framing, and the internal FIFO buffering for both TX and RX.

\renewcommand*{\arraystretch}{1.3}
\begingroup
\small
\rowcolors{2}{gray!30}{gray!10}
\arrayrulecolor{gray!80}

\begin{longtable}[H]{
  | p{0.22\textwidth}
  | p{0.13\textwidth}
  | p{0.08\textwidth}
  | p{0.08\textwidth}
  | p{0.42\textwidth} |
}
\hline
\rowcolor{gray}
\textcolor{white}{\textbf{Name}} &
\textcolor{white}{\textbf{Type}} &
\textcolor{white}{\textbf{Min}} &
\textcolor{white}{\textbf{Max}} &
\textcolor{white}{\textbf{Description}} \\ 
\hline
\endfirsthead

\hline
\rowcolor{gray}
\textcolor{white}{\textbf{Name}} &
\textcolor{white}{\textbf{Type}} &
\textcolor{white}{\textbf{Min}} &
\textcolor{white}{\textbf{Max}} &
\textcolor{white}{\textbf{Description}} \\ 
\hline
\endhead

\hline
\endfoot

\texttt{dataWidth} &
Integer &
8 &
32 &
Data width for APB bus operations. \\ \hline

\texttt{addressWidth} &
Integer &
1 &
-- &
Address width for APB bus. \\ \hline

\texttt{maxClockFrequency} &
Integer &
-- &
-- &
Maximum system clock frequency in Hz (used by the internal Baud Rate Generator). \\ \hline

\texttt{maxBaudRate} &
Integer &
-- &
-- &
Maximum achievable baud rate (used to size internal counters). \\ \hline

\texttt{maxOutputBits} &
Integer &
5 &
16 &
Maximum number of data bits plus optional parity bit (TX/RX). \\ \hline

\texttt{syncDepth} &
Integer &
2 &
-- &
Depth for RX input synchronization flip‐flops. \\ \hline

\texttt{parity} &
Bool &
N/A &
N/A &
Default parity selection (odd/even) at reset, may be overridden in registers. \\ \hline

\texttt{verbose} &
Bool &
N/A &
N/A &
Enables debug \texttt{printf} statements. \\ \hline

\textbf{\texttt{bufferSize}} &
Integer &
1 &
-- &
Depth of the \textit{DynamicFifo} used in TX and RX. Must be a power of 2 for certain synthesis flows. \\ \hline

\end{longtable}
\captionof{table}{UART Parameter Descriptions}
\label{table:uart_params}
\endgroup

\noindent
\textbf{Note:} The new \texttt{bufferSize} parameter defines how many words/bytes can be queued in the internal FIFOs for TX and RX. If set to 1, you effectively have a minimal buffer. Larger depths (e.g.\ 16, 32) allow more data to be queued without CPU intervention.
