\section{Synthesis}

\subsection{Area and Gate Count}
Example synthesis results vary based on \texttt{fifoDepth} and other settings. An illustrative table:

\renewcommand*{\arraystretch}{1.3}
\begingroup
\small
\rowcolors{2}{gray!30}{gray!10}
\begin{longtable}[H]{
  | p{0.22\textwidth}
  | p{0.18\textwidth}
  | p{0.20\textwidth}
  | p{0.30\textwidth} |
}
\hline
\rowcolor{gray}
\textcolor{white}{\textbf{Config}} &
\textcolor{white}{\textbf{FIFO Depth}} &
\textcolor{white}{\textbf{Data Bits}} &
\textcolor{white}{\textbf{Gate Count}} \\ 
\hline
\endfirsthead

\hline
\rowcolor{gray}
\textcolor{white}{\textbf{Config}} &
\textcolor{white}{\textbf{FIFO Depth}} &
\textcolor{white}{\textbf{Data Bits}} &
\textcolor{white}{\textbf{Gate Count}} \\ 
\hline
\endhead

\hline
\endfoot

\texttt{uart\_8\_depth4}  & 4  & 8  & 2,200 gates \\ \hline
\texttt{uart\_8\_depth16} & 16 & 8  & 3,100 gates \\ \hline
\texttt{uart\_9\_depth16} & 16 & 9  & 3,400 gates \\ \hline

\caption{Example Synthesis Results}
\end{longtable}
\endgroup

\subsection{Timing Constraints}
An \texttt{.sdc} file can be generated to specify clock period, input/output constraints, etc. For example:
\begin{verbatim}
create_clock -name PCLK -period 10.0 [get_ports PCLK]
set_input_delay 2.0 -clock PCLK [get_ports rx]
set_output_delay 2.0 -clock PCLK [get_ports tx]
\end{verbatim}

\subsection{Multicycle and False Paths}
Typically, no multicycle or false paths are required if the design runs entirely from \texttt{PCLK} with no asynchronous logic. Check your usage and tool warnings for final closure.
