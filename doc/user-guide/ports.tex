\section{Port Descriptions}

\subsection{UART Interface}

Table~\ref{table:ports} shows the top-level ports related to the UART signals.

\renewcommand*{\arraystretch}{1.3}
\begingroup
\small
\rowcolors{2}{gray!30}{gray!10}
\arrayrulecolor{gray!80}

\begin{longtable}[H]{
  | p{0.20\textwidth}
  | p{0.20\textwidth}
  | p{0.10\textwidth}
  | p{0.40\textwidth} |
}
\hline
\rowcolor{gray}
\textcolor{white}{\textbf{Port Name}} &
\textcolor{white}{\textbf{Width}} &
\textcolor{white}{\textbf{Direction}} &
\textcolor{white}{\textbf{Description}} \\ 
\hline
\endfirsthead

\hline
\rowcolor{gray}
\textcolor{white}{\textbf{Port Name}} &
\textcolor{white}{\textbf{Width}} &
\textcolor{white}{\textbf{Direction}} &
\textcolor{white}{\textbf{Description}} \\ 
\hline
\endhead

\hline
\endfoot

\texttt{rx} &
1 &
Input &
Receive data line (serial input). \\ \hline

\texttt{tx} &
1 &
Output &
Transmit data line (serial output). \\ \hline

\end{longtable}
\captionof{table}{UART-Specific I/O Ports}
\label{table:ports}
\endgroup

\subsection{APB3 Interface}

The \textbf{APB3 Interface} is a typical APB Slave interface. See Table~\ref{table:uart_apb_ports} for details. The widths of the data/address buses are parameterized by \textit{dataWidth} and \textit{addressWidth}.

\renewcommand*{\arraystretch}{1.3}
\begingroup
\small
\rowcolors{2}{gray!30}{gray!10}
\arrayrulecolor{gray!80}

\begin{longtable}[H]{
  | p{0.20\textwidth}
  | p{0.20\textwidth}
  | p{0.10\textwidth}
  | p{0.40\textwidth} |
}
\hline
\rowcolor{gray}
\textcolor{white}{\textbf{Port Name}} &
\textcolor{white}{\textbf{Width}} &
\textcolor{white}{\textbf{Direction}} &
\textcolor{white}{\textbf{Description}} \\ 
\hline
\endfirsthead

\hline
\rowcolor{gray}
\textcolor{white}{\textbf{Port Name}} &
\textcolor{white}{\textbf{Width}} &
\textcolor{white}{\textbf{Direction}} &
\textcolor{white}{\textbf{Description}}\\ 
\hline
\endhead

\hline
\endfoot

\texttt{PCLK} &
1 &
Input &
Positive edge clock for register accesses. \\ \hline

\texttt{PRESETN} &
1 &
Input &
Active-low asynchronous reset. \\ \hline

\texttt{PSEL} &
1 &
Input &
Select signal for APB slave. \\ \hline

\texttt{PENABLE} &
1 &
Input &
Indicates second cycle of APB transfer. \\ \hline

\texttt{PWRITE} &
1 &
Input &
\texttt{HIGH} for write, \texttt{LOW} for read. \\ \hline

\texttt{PADDR} &
\textit{addressWidth} &
Input &
APB address bus. \\ \hline

\texttt{PWDATA} &
\textit{dataWidth} &
Input &
APB write data bus. \\ \hline

\texttt{PRDATA} &
\textit{dataWidth} &
Output &
APB read data bus. \\ \hline

\texttt{PREADY} &
1 &
Output &
Indicates transfer ready. \\ \hline

\texttt{PSLVERR} &
1 &
Output &
Indicates a transfer error. \\ \hline

\end{longtable}
\captionof{table}{APB Ports Descriptions}
\label{table:uart_apb_ports}
\endgroup
