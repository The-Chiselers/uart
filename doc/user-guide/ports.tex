\section{Port Descriptions}

The UART module provides two sets of ports:
\begin{enumerate}
  \item The external UART interface (serial TX and RX signals)
  \item The APB interface for register access
\end{enumerate}

\subsection{UART Interface}

The UART-specific ports allow the module to communicate with external devices over a standard asynchronous serial link. The key signals are:

\renewcommand*{\arraystretch}{1.3}
\begingroup
\small
\rowcolors{2}{gray!30}{gray!10}
\begin{longtable}[H]{
  | p{0.25\textwidth}
  | p{0.15\textwidth}
  | p{0.15\textwidth}
  | p{0.40\textwidth} |
}
\hline
\rowcolor{gray}
\textcolor{white}{\textbf{Port Name}} &
\textcolor{white}{\textbf{Width}} &
\textcolor{white}{\textbf{Direction}} &
\textcolor{white}{\textbf{Description}} \\ \hline
\endfirsthead

\hline
\rowcolor{gray}
\textcolor{white}{\textbf{Port Name}} &
\textcolor{white}{\textbf{Width}} &
\textcolor{white}{\textbf{Direction}} &
\textcolor{white}{\textbf{Description}} \\ \hline
\endhead

\hline
\endfoot

\texttt{rx} &
1 &
Input &
Asynchronous serial receive data input. This signal is synchronized internally and sampled by the receiver FSM. \\ \hline

\texttt{tx} &
1 &
Output &
Serial transmit data output. When idle, this line remains high. \\ \hline
\end{longtable}
\captionof{table}{UART Interface Ports}
\label{table:uart_ports}
\endgroup

\subsection{APB Interface}

The APB interface provides access to the UART’s registers, allowing software to configure and monitor both the TX and RX operations (including baud rate settings, FIFO control, and error status). The APB signals are defined as follows:

\renewcommand*{\arraystretch}{1.3}
\begingroup
\small
\rowcolors{2}{gray!30}{gray!10}
\begin{longtable}[H]{
  | p{0.20\textwidth}
  | p{0.20\textwidth}
  | p{0.12\textwidth}
  | p{0.43\textwidth} |
}
\hline
\rowcolor{gray}
\textcolor{white}{\textbf{Port Name}} &
\textcolor{white}{\textbf{Width}} &
\textcolor{white}{\textbf{Direction}} &
\textcolor{white}{\textbf{Description}} \\ \hline
\endfirsthead

\hline
\rowcolor{gray}
\textcolor{white}{\textbf{Port Name}} &
\textcolor{white}{\textbf{Width}} &
\textcolor{white}{\textbf{Direction}} &
\textcolor{white}{\textbf{Description}}\\ \hline
\endhead

\hline
\endfoot

\texttt{PCLK} &
1 &
Input &
APB clock signal for register access. \\ \hline

\texttt{PRESETN} &
1 &
Input &
Active–low asynchronous reset. \\ \hline

\texttt{PSEL} &
1 &
Input &
Select signal indicating that the UART is addressed. \\ \hline

\texttt{PENABLE} &
1 &
Input &
Indicates the second cycle of an APB transfer. \\ \hline

\texttt{PWRITE} &
1 &
Input &
Determines the operation: HIGH for write, LOW for read. \\ \hline

\texttt{PADDR} &
\textit{addressWidth} &
Input &
APB address bus for register selection. \\ \hline

\texttt{PWDATA} &
\textit{dataWidth} &
Input &
APB write data bus. \\ \hline

\texttt{PRDATA} &
\textit{dataWidth} &
Output &
APB read data bus. \\ \hline

\texttt{PREADY} &
1 &
Output &
Indicates that the UART is ready for the next transfer. \\ \hline

\texttt{PSLVERR} &
1 &
Output &
Indicates a transfer error. \\ \hline
\end{longtable}
\captionof{table}{APB Interface Ports}
\label{table:apb_ports}
\endgroup
