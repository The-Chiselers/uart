\section{Port Descriptions}

The ports for \textbf{Timer} are shown below in Table~\ref{table:ports}. The width of several ports is controlled by the following input parameters:

\begin{itemize}[noitemsep]
  \item \textit{dataWidth} is the width of the data bus in bits.
  \item \textit{addressWidth} is the width of the address bus in bits.
  \item \textit{countWidth} is the width of the counter in bits.
\end{itemize}

\renewcommand*{\arraystretch}{1.4}
\begin{longtable}[H]{
  | p{0.20\textwidth}
  | p{0.20\textwidth}
  | p{0.12\textwidth}
  | p{0.43\textwidth} |
  }
  \hline
  \textbf{Port Name} &
  \textbf{Width} &
  \textbf{Direction} &
  \textbf{Description} \\ \hline \hline

  clock &
  1 &
  Input &
  Positive edge clock \\ \hline

  reset &
  1 &
  Input &
  Active high reset \\ \hline

  en &
  1 &
  Input &
  Enable signal for the timer \\ \hline

  prescaler &
  \textit{countWidth} &
  Input &
  Prescaler value to divide the clock frequency \\ \hline

  maxCount &
  \textit{countWidth} &
  Input &
  Maximum count value before the timer resets \\ \hline

  pwmCeiling &
  \textit{countWidth} &
  Input &
  PWM ceiling value to control the duty cycle of the PWM signal \\ \hline

  setCountValue &
  \textit{countWidth} &
  Input &
  Value to set the counter to when \textit{setCount} is asserted \\ \hline

  setCount &
  1 &
  Input &
  Signal to set the counter to \textit{setCountValue} \\ \hline

  count &
  \textit{countWidth} &
  Output &
  Current count value of the timer \\ \hline

  maxReached &
  1 &
  Output &
  Signal indicating that the timer has reached its maximum count value \\ \hline

  pwm &
  1 &
  Output &
  PWM output signal with a duty cycle controlled by \textit{pwmCeiling} \\ \hline

  interrupt &
  1 &
  Output &
  Interrupt signal indicating timer events (e.g., max reached) \\ \hline

  \caption{Port Descriptions}\label{table:ports}
\end{longtable}